
\documentclass{article} % For LaTeX2e
\usepackage{iclr2024_conference,times}

% Optional math commands from https://github.com/goodfeli/dlbook_notation.
\input{math_commands.tex}

\usepackage{hyperref}
\usepackage{url}


\title{InformatiCup 2024 Ctrl-Alt-Defeat}

% Authors must not appear in the submitted version. They should be hidden
% as long as the \iclrfinalcopy macro remains commented out below.
% Non-anonymous submissions will be rejected without review.

\author{Max Buchholz, Lukas Müller, Levi Otterbach \& Raphael Weber}

% The \author macro works with any number of authors. There are two commands
% used to separate the names and addresses of multiple authors: \And and \AND.
%
% Using \And between authors leaves it to \LaTeX{} to determine where to break
% the lines. Using \AND forces a linebreak at that point. So, if \LaTeX{}
% puts 3 of 4 authors names on the first line, and the last on the second
% line, try using \AND instead of \And before the third author name.

\newcommand{\fix}{\marginpar{FIX}}
\newcommand{\new}{\marginpar{NEW}}

\iclrfinalcopy % Uncomment for camera-ready version, but NOT for submission.
\begin{document}


\maketitle

\begin{abstract}

\end{abstract}

\section{Introdution}

\section{Related work}

\subsection{Generating models}

\subsubsection{Image generating models}

\subsubsection{Text generating models}

\subsection{Detector models}

\subsubsection{Image detector models}

\subsubsection{Text detector models}

\section{Method}
To test and evaluate the effect of different data augmentation techniques on the detection capabilities of fake image or text detectors, we created a pipeline of three components: The first component is a generator, which is a model capable of generating fake images or texts. The second component is a processor which augments the data through a specified method, after inputting a generated image or text. The last component is a detector, which suggests an input being either real or fake. This way we can compare it to the ground truth, to evaluate the capabilities of the detector. By comparing the results with augmented artefacts and with the generated ones without augmentation, we can evaluate the effectiveness of different AI detector-tricking techniques.

\section{Results}

\section{Discussion}

\section{Conclusion}


\subsubsection*{Acknowledgments}


\bibliography{bibliography}
\bibliographystyle{iclr2024_conference}

\appendix
\section{Appendix}
You may include other additional sections here.

\end{document}
